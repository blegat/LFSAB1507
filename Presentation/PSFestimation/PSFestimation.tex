\section[PSF Estimat.]{PSF Estimation}
\subsection{Angle estimation with Radon}
\begin{frame}[allowframebreaks]
  \frametitle{Effect of the Blur on the Fourier transform}
  \begin{align*}
    g & = h * f
    & G &= H \cdot F
    & S & = \log(1 + |G|).
  \end{align*}
  Problem at \ang{0} and \ang{90} due to non-periodicity of the image.
  \begin{myfig}{cameraman_F_diff}
    {Effect of the blur on the fourier transform of the cameraman}
    \mysubfig{cameraman_F}{$S$ with no blur.}{0.32}
    \mysubfig{cameraman_F_replicate_40-30}{$S$ with ``replicate''.}{0.32}
    \mysubfig{cameraman_F_circular_40-30}{$S$ with ``circular''.}{0.32}
  \end{myfig}

  \framebreak

  \begin{enumerate}
    \item Square it.
    \item Apply hann window in spatial domain.
  \end{enumerate}
  \begin{myfig}{sagar-hann}
    {Hann window for Sagar which has been blurred at $(25,\ang{171})$.}
    \mysubfig{hann}{Hann window.}{0.3}
    \mysubfig{sagar-hann}{Squared and ``hanned''.}{0.35}
    \mysubfig{sagar-S}{$S$ with a clear $\sinc$.}{0.3}
  \end{myfig}
\end{frame}

\begin{frame}[allowframebreaks]
  \frametitle{Angle estimation with Radon}
  \begin{block}{Radon transform (RT)}
    \[ R_\theta(x') = \int_{y_0'(x')}^{y_1'(x')} f'(x',y') \dif y' \]
    where frame $(x',y')$ is $(x,y)$ rotated by $\theta$ counterclockwise.
  \end{block}

  \begin{myfig}{ones}
    {Analysis of the RT on an unit matrix (filled with ones)}
    \mysubfig{ones-Rad}{RT of an unit matrix.}{0.35}
    \mysubfig{ones-Var}{Variance and max for each angle of the RT.}{0.35}
  \end{myfig}

  \begin{myfig}{sagar}
    {RT for Sagar which has been blurred at $(25,\ang{171})$.}
    \mysubfigg{sagar-Rad}{RT of $S$.}{0.3}{trim=8cm 0cm 8cm 0cm, clip, height=4cm}
    \mysubfigg{sagar-RadNorm}{Normalized RT of $S$.}{0.3}{trim=8cm 0cm 8cm 0cm, clip, height=4cm}
    \mysubfigg{sagar-RadFocus}{Normalized and cropped RT of $S$.}{0.3}{trim=7cm 0cm 7cm 0cm, clip, height=4cm}
  \end{myfig}

  \begin{myfig}{sagar-plot}
    {$\var$ and $\max$ of the RT of $S$ for Sagar}
    \mysubfig{sagar-Var}{$\var$ of the RT.}{0.28}%
    \mysubfig{sagar-VarN}{$\var$ of the normalized RT.}{0.28}
    \mysubfig{sagar-max}{$\max$ of the RT.}{0.28}%
    \mysubfig{sagar-maxN}{$\max$ of the normalized RT.}{0.28}
  \end{myfig}
\end{frame}

\subsection{Angle estimation with Gabor}
\begin{frame}
	\frametitle{Angle estimation with Gabor}
	steps:
	\begin{enumerate}
	\item spectrum of the blurred image
	\item logarithm of the spectrum, i.e. $I=\log(G(u,v))$, is used as input for Gabor filter
	\item Gabor filter with different orientations ($\phi$) are convolved with $I$ to get different responses $R(\phi)$
	\item for every $\phi$, take the $L_2$ norm
	\item blur angle : $\alpha = arg{max_{\phi}||R(\phi)||_2}$
	\end{enumerate}
	
	Problem: didn't work, not enough time to find errors
\end{frame}


\subsection{Length estimation with modified Cepstrum}
\begin{frame}[allowframebreaks]
  \frametitle{Length estimation with modified Cepstrum}
  Let $S = \log(|G|)$
  \begin{block}{Classical Cepstrum}
    \[ \F^{-1}(S) = 
    \F^{-1}(\log(|H|)) + \F^{-1}(\log(|F|)) \]
  \end{block}
  \begin{block}{Modified Cepstrum}
    \[ \log(\F(|S|)). \]
  \end{block}
  \begin{myfig}{cameraman_cepstrum}
    {Modified Ceptstrum on the cameraman blurred with $L = 60$.}
    \mysubfig{cameraman-plot-60_0}{Plot of the horizontal line for \ang{0}}{0.28}%
    \mysubfig{cameraman-contourf-60_0}{Modified Cepstrum for \ang{0}}{0.28}
    \mysubfig{cameraman-plot-60_30}{Plot of the horizontal line for \ang{30}}{0.28}%
    \mysubfig{cameraman-contourf-60_30}{Modified Cepstrum for \ang{30}}{0.28}
  \end{myfig}
\end{frame}


\NeedsTeXFormat{LaTeX2e}

\usetheme[compress]{Singapore} % theme

\usepackage[frenchb]{babel}
\usepackage[T1]{fontenc}
\usepackage[utf8x]{inputenc}
\usepackage{lmodern}
\usepackage{amsmath,amsthm,amssymb}        % un packages mathématiques
\usepackage{xcolor}         % pour définir plus de couleurs
\usepackage{graphicx}       % pour insérer des figures
\graphicspath{{../Images/}}
\usepackage{lmodern}
\usepackage{lastpage}
\usepackage{endnotes}
\usepackage{hyperref}

\usepackage{todonotes}


\usepackage{listings}
\usepackage{listingsutf8}

\usepackage{siunitx}
\usepackage{wrapfig}
\usepackage{pdfpages}
\usepackage{verbatim}



\usepackage{xparse}% for using parameters at the end block

\NewDocumentEnvironment{myfig}{mm}
{\begin{figure}\centering}
{\caption{#2}\label{fig:#1}\end{figure}}

\NewDocumentEnvironment{myfigsub}{mmm}
{\begin{subfigure}[b]{#3\textwidth}}
{\caption{#2}\label{fig:#1}\end{subfigure}}

\newcommand{\mysubfig}[3]
{\begin{myfigsub}{#1}{#2}{#3}
    \includegraphics[width=\textwidth]{#1.png}
\end{myfigsub}}
\newcommand{\mysubfigg}[4]
{\begin{myfigsub}{#1}{#2}{#3}
    \includegraphics[#4, width=\textwidth]{#1.png}
\end{myfigsub}}


\newcommand{\myfullfig}[3]
{\begin{figure}[!ht]
    \centering
    \includegraphics[width=#3\textwidth]{#1.png}
    \caption{#2}
    \label{fig:#1}
\end{figure}}



\DeclareFontFamily{OT1}{pzc}{}
\DeclareFontShape{OT1}{pzc}{m}{it}{<-> s * [1.10] pzcmi7t}{}
\DeclareMathAlphabet{\mathpzc}{OT1}{pzc}{m}{it}

\DeclareMathOperator{\newdiff}{d} % use \dif instead
\newcommand{\dif}{\newdiff\!}
\newcommand{\fpart}[2]{\frac{\partial #1}{\partial #2}}
\newcommand{\ffpart}[2]{\frac{\partial^2 #1}{\partial #2^2}}
\newcommand{\fdpart}[3]{\frac{\partial^2 #1}{\partial #2\partial #3}}
\newcommand{\fdif}[2]{\frac{\dif #1}{\dif #2}}
\newcommand{\ffdif}[2]{\frac{\dif^2 #1}{\dif #2^2}}
\newcommand{\constant}{\ensuremath{\mathrm{cst}}}
\newcommand{\rha}{\hat{r}^n_{MLE}}
\newcommand{\bigoh}{\mathcal{O}}
\newcommand{\F}{\mathcal{F}}

\DeclareMathOperator{\pois}{Pois}
\DeclareMathOperator{\sinc}{sinc}
\DeclareMathOperator{\var}{Var}
\DeclareMathOperator{\argmax}{argmax}

\usepackage{parskip} % Ajoute de l'espace entre les paragraphes et mets l'indentation to 0
\setlength{\parindent}{15pt} % Remets l'indentation par default

\newcommand{\figref}[1]{figure~\ref{fig:#1}}

%\usepackage[svgnames]{color}
%\definecolor{webdarkblue}{rgb}{0,0,0.4}
%\definecolor{webgreen}{rgb}{0,0.3,0}
%\definecolor{webblue}{rgb}{0,0,0.8}

\setbeamercolor{section in head/foot}{use=structure,bg=structure.fg!25!bg} % "Amélioration du jeu de couleur"
%\useoutertheme[subsection=true]{smoothbars} % Pour avoir un rappel de la subsection
\setbeamerfont{frametitle}{series=\bfseries}
\setbeamertemplate{frametitle}[default][center] % Titre centré et bien placé.


% "Fioriture de style" : qd <x-> dans les item, les autres en gris clair
\beamertemplatetransparentcovered


% Comportement des itemize
\setbeamertemplate{itemize item}[ball]
\setbeamertemplate{itemize subitem}[triangle]
\setbeamertemplate{itemize subsubitem}[circle]

%\renewcommand\sfdefault{cmss} % Polices

% Les block arrondis et ombrés dans la couleur que je veux
\setbeamertemplate{blocks}[rounded][shadow=true]
\definecolor{normalBlockColor}{RGB}{255,255,255}
\definecolor{normalTitleBlockColor}{RGB}{0,0,102}
\definecolor{normalBlockTextColor}{RGB}{0,0,0}
\definecolor{normalBlockTitleTextColor}{RGB}{255,255,255}
\definecolor{exampleBlockColor}{RGB}{202,251,197}
\definecolor{exampleTitleBlockColor}{RGB}{166,241,158}
\definecolor{exampleBlockTextColor}{RGB}{0,0,0}
\definecolor{exampleBlockTitleTextColor}{RGB}{0,120,0}
\definecolor{alertBlockColor}{RGB}{248,218,218}
\definecolor{alertTitleBlockColor}{RGB}{244,108,108}
\definecolor{alertBlockTextColor}{RGB}{0,0,0}
\definecolor{alertBlockTitleTextColor}{RGB}{120,0,0}
\setbeamercolor*{block title}{fg=normalBlockTitleTextColor,bg=normalTitleBlockColor}
\setbeamercolor*{block body}{fg=normalBlockTextColor,bg=normalBlockColor}
\setbeamercolor*{block title alerted}{fg=alertBlockTitleTextColor,bg=alertTitleBlockColor}
\setbeamercolor*{block body alerted}{fg=alertBlockTextColor,bg=alertBlockColor}
\setbeamercolor*{block title example}{fg=exampleBlockTitleTextColor,bg=exampleTitleBlockColor}
\setbeamercolor*{block body example}{fg=exampleBlockTextColor,bg=exampleBlockColor}
\setbeamerfont{block title}{size={}}



%------------ fin style beamer -------------------

% Faire apparaître un sommaire avant chaque section
% \AtBeginSection[]{
%   \begin{frame}
%   \frametitle{Plan}
%   \medskip
%   %%% affiche en début de chaque section, les noms de sections et
%   %%% noms de sous-sections de la section en cours.
%   \small \tableofcontents[currentsection, hideothersubsections]
%   \end{frame}
% }


% Pour personnaliser la barre de navigation du dessous
\setbeamertemplate{navigation symbols}{
	%\insertslidenavigationsymbol
	%\insertframenavigationsymbol
	%\insertsubsectionnavigationsymbol
	\quad\textbf{\insertframenumber/\inserttotalframenumber} % Numéro de page
	%\insertsectionnavigationsymbol
	%\insertdocnavigationsymbol
	%\insertbackfindforwardnavigationsymbol
}
% Supprimer les icones de navigation (pour les transparents)
%\setbeamertemplate{navigation symbols}{}

% Mettre les icones de navigation en mode vertical (pour projection)
% \setbeamertemplate{navigation symbols}[vertical]

%\newenvironment{itemize2}%
%	{ \begin{list}%
%		{$\bullet$}%
%		{\setlength{\labelwidth}{30pt}%
%		 \setlength{\leftmargin}{35pt}%
%		 \setlength{\itemsep}{\parsep}}}%
%	{ \end{list} }

%\def\siecle#1{\textsc{\romannumeral #1}\textsuperscript{e}~siècle} % => le \siecle{19}

%\definecolor{codeBlue}{rgb}{0,0,1}
%\definecolor{webred}{rgb}{0.5,0,0}
%\definecolor{codeGreen}{rgb}{0,0.5,0}
%\definecolor{codeGrey}{rgb}{0.6,0.6,0.6}
%\definecolor{webdarkblue}{rgb}{0,0,0.4}
%\definecolor{webgreen}{rgb}{0,0.3,0}
%\definecolor{webblue}{rgb}{0,0,0.8}
%\definecolor{orange}{rgb}{0.7,0.1,0.1}
%\lstset{
%      language=TeX,
%      flexiblecolumns=true,
%      numbers=left,
%      stepnumber=1,
%      numberstyle=\ttfamily\tiny,
%      keywordstyle=\ttfamily\textcolor{blue},
%      stringstyle=\ttfamily\textcolor{red},
%      commentstyle=\ttfamily\textcolor{codeGreen},
%      breaklines=true,
%      extendedchars=true,
%      basicstyle=\ttfamily\scriptsize,
%      showstringspaces=false,
%      morekeywords={usepackage,documentclass,begin,textbf,textit,texttt,ref,includegraphics,caption,label,setlength,mathbb,notag,frac,num,si,ang,SI,textwidth,percent,meter,ohm,joule,second,more,section,subsection,tableofcontents,setstretch,TeX,LaTeX,huge,sffamily,emph,chemfig,pageref,vpageref},
%      frame=single,
%      extendedchars=true,
%      inputencoding=utf8x
%    }
%\lstset{inputencoding=utf8/latin1}

\renewcommand{\subsection}[1]{}

%TODO : expliquer quelque part le paramètre de length estimator (3, 5 ou 8)

\section{Comparison and interpretation of results for the train}

The purpose of this section is to interpret and compare results on different pictures deblurred by the three methods of deconvolution used: Lucy-Richardson, Wiener and regularization. We will test essentially ideal images artificially blurred in order to isolate the influence of certain parameters. Although most of the images used are not taken inside of a train as it indicated, the assumption of a linear motion blur is well respected and we are therefore in the first case treated (the train).

\subsection{Noise influence}

Nous nous intéressons ici au défloutage d'image de départ avec bruit (indépendant du signal). Trois cas différents sont envisagés: l'image de départ sans bruit, avec bruit Gaussien de moyenne nulle et de variance $0.01$ et avec bruit speckle (très léger). L'image originale n'a pas été floutée artificiellement et les photos défloutées ont été compressées pendant le traitement (voir section [?????]). Le bruit a été inséré par des fonctions prédéfinies Matlab. Ci-dessous les résultats obtenus:

%TODO : mettre les images du dossier ../Images/Results/desert noise pour comparer les résultat obtenus. Je pensais mettre les images disposées en rectangle 4*3 où à l'horizontal on à les 3 cas (de gauche à droite : sans bruit, avec Gaussien et avec speckle) et en vertical l'image original + les 3 images défloutées (donc de haut en bas : image originale, Lucy, Wiener, Reguralisation). 

Nous remarquons que l'algorithme de reguralisation est nettement plus sensible à la présence de bruit dans l'image initiale, pour les deux type de bruits injectés au départ. Les deux autres méthodes sont également influencée mais le défloutage n'est pas compromis (pour autant que le bruit initial n'est pas trop important).

%TODO : possible d'expliquer ça mathématiquement ? 

\subsection{influence de la longueur de floutage estimée}

Pour estimer la PSF, il est nécessaire d'estimer le paramètre de longueur de floutage (voir section [????]). Il est alors intéressant d'étudier l'impact de ce facteur sur la photo de sortie et d'établir la sensibilité des trois méthodes utilisées. 
L'image en exemple est la célèbre photo prise en 1972 de Lena, playmate du magazine playboy et est souvent utilisée comme test pour les algorithmes de traitement d'image (REFERENCE VERS WIKIPEDIA). Celle-ci a été floutée artificiellement ici selon un angle de 20 degrés et une longueur de 30 pixels. LE principe a été de fixer la longueur estimée à une certaine valeur afin de voir l'influence qu'à cette valeur sur le résultat final (il est à noter que l'angle estimé par l'algorithme est 20 degrés ce qui est exact et le paramètre de longueur est donc la seule variable ici). Pour chaque méthode, trois cas ont été traités: la longueur estimée (que nous fixons) est de 20 pixels, de 30 pixels (valeur exacte) et de 40 pixels. Ci-dessous les résultats obtenus : 

%TODO dans ../Images/Results/Lena/Blur20deg30length. je pensais mettre d'abord l'image originale et l'image floutée artificiellement côte à cote puis en dessous rectangle 3*3 avec (notation matrice matlab) : [L20 L30 L40; W20 W30 W40; R20 R30 R40]

Il ressort directement que la méthode de regularisation est bien plus sensible au paramètre étudié que les deux autres méthodes. En effet, des artefacts apparaissent plus le paramètre estimé s'éloigne de la valeur exacte. L'exemple suivant le montre bien (plaque d'immatriculation floutée artificiellement à 0 degré et 40 pixels) : 

%TODO dans ../Images/Results/plaque/Blur0angle40length. mettre d'abord plaque originale (originalimage) à côté de ArtificialBlur puis après [RegL38 RegL39 RegL40; RegL41 RegL42]

%TODO Y a moyen d'expliquer pourquoi reg est plus sensible mathématiquement vous pensez ? 

\subsection{influence de la compression sur la vitesse d'exécution}

\subsection{comparaison des temps d'exécution}

\subsection{influence du paramètre de length estimator}

%TODO Arnaud comme on avait parlé hier et l'influence du paramètre (disons P) qui vaut 3, 5 ou 8. Moi je comprend pas trop donc ca serait pas mal si on arrive à expliquer ça avec le graphique des pics et tout. ..\Images\Results\plaque\55 y a une la plaque floutée artificiellement à 55 pixels (et angle 0) et quand on prend que P = 3 ou 5 (1 ou 2 dans le GUI), ça estime la longueur à 23 (à la place de 55), du coup ça donne Lucy_P1 Wiener_P1 et Reg_P1 (de la merde...). Tandis que P = 8 (3 dans le GUI) estime la longueur à 54, et ça donne les beaux résultats Lucy_P3 Wiener_P3 et Reg_P3. Tu sais en tirer quelque chose ? 

\subsection{influence des couleurs}
\subsection{images réelles}
\subsection{conclusion}
\section{blur metrics}

To compare different methods of deblurring, we need to be able to give the perceptual quality of an image. Suppose we have the original image $F$ and we want to estimate the quality of another image $\hat{F}$ that is a blurred version of $F$ or an approximation of $F$ after a deblurring operation. An easy way to do this is computing the Mean Square Error (MSE) between both images:
\begin{equation}
MSE=\frac{1}{MN} \sum\limits_{m=1}^{n=1}\left(F(m,n)-\hat{F}(m,n)\right)^2.
\end{equation}

The problem of this method is that it highly depends on the scale of intensity of the image. That's why we prefer the Peak Signal-to-Noise Ratio (PSNR) that avoid this problem:
\begin{equation}
PSNR = -10 log \frac{MSE}{S^2},
\end{equation}
where $S$ is the maximum value that a pixel can reach (255 for 8bits coding). Here the bigger the PSNR is, the better the quality of the image.

The problem of both methods is that we need the original image. That is only possible when we blur the images ourselves and we test our different deblurring methods. But in a real situation, we don't know $F$. The only image we have is the blurred image $G$ and the result of a deblurring method $\hat{F}$, so we can apply MSE and PSNR to those images. But we still only get an estimation of how close both images are, not of the blur intensity itself. Our next method determines the quality of an image objectively and without referential image. The technique that we found is a no-reference blur metric that is based on the analysis of the spread of the edges in an image. No-reference means that the metric is not relative to the original but is a absolute value associated to an image.

When an image is blurred, its high spatial frequency values in the spectrum are attenuated. Blur is perceptually apparent along edges so the technique is based on the smoothing effect of it on the edges. According to SOOOOOURCE (marziliano2002),it is sufficient to measure blur along vertical edgdes. Here is the algorithm:\\
First we detect the edges by a Sobel filter for example. Then we scan each row of the image. When we detect a pixel that corresponds to an edge, we look for the local maximum and minimum on that row. The distance (in pixels) between those extrema represents the local blur. Finally, we get the global blur measure for whole the image by averaging the local blur values over all edge locations.

The matlab function $bordSobel(I,meth,tresh,direction)$ implements this algorithm. The inputs are the processed image $I$, the method used to determine the edges ($1=Sobel, 2=Prewitt, 3=Canny$), a treshold for those methods that is fixed at $0.0215$ if it is not specified and finally the direction of the edges (vertical or horizontal).

%TODO résultats
%TODO changer nom de bordSobel car ca peut une autre meth que Sobel




\section{Complementary Treatment}
\subsection{EdgeTaper}

\subsection{RGB - grayscaled images : analysis and treatment}

Comme expliqué dans la partie de modélisation mathématique, les images NB se distinguent aux images de couleurs dans leur structure mathématique. La façon pour matlab de gérer l'espace des couleurs est d'adopter une représentation RGB de telle sorte que l'output pour un certain point du plan est un vecteur à 3 dimensions (one dimension per primitive colour) au lieu d'un scalaire dans le cas d'une image NB. L'image est donc représentée par une matrice à 3 dimensions et la couleur finale en un point correspond à une synthèse additive de ces trois composantes.

Dans le cas de notre travail, nous avons adapté le programme de telle sorte à ce qu'il fonctionne pour les deux types d'images (RGB ou NB). Il est également possible pour un utilisateur de choisir de convertir l'image à traiter en NB via l'interface graphique (qui exploite la fonction rgbtogray.m). L'avantage de convertir une image en NB avant de la traiter est le gain de temps (une dimension à traiter au lien des 3 de RGB). Par exemple le gain de temps en convertissant les images de la camera de surveillance peut être assez important. De plus, suivant l'utilisation voulue, il n'est pas toujours nécessaire d'avoir une image finale en couleur (par exemple pour le défloutage d'une plaque d'immatriculation). 
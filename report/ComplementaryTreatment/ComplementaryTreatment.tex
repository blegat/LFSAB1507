\section{Complementary Treatment}

\subsection{Estimation of the noise-to-signal power ratio (NSR)}
\label{subsec:NSREstimation}
Deconvolution by the method of Wiener needs the NSR of the additive noise in the image. The problem of computing the NSR is that we need the original image, which we usually don't have, otherwise deblurring would have no meaning anymore. That is why we try to estimate the SNR by the following reasoning. The pixels of a zone that represent a background (e.g. an area of a blue sky) should be approximatively constant. So the variance of this area would me mostly due to the noise. So we will estimate the power of noise by the variance of such an area. This is the main idea of our matlab function $nsrEstimation(f,psf)$, with $f$ the processed image. % ca sert à quoi psf? %cmt choisir intervention humaine ou pas? if true..?

This function proposes two ways to handle this. In the first one, the user must select an area that he considers as background. In the second way, the function selects automatically five regions: four in the corners and one in the center. It computes the variance for each region and it takes the zone with the least variance. The second method is probably less precise because the predefined regions might not represent parts of a background. But it has the advantage that it doesn't need a human intervention which means there is less work for the user and deblurring also goes faster. Indeed, it's not possible to ask the user to detect a region of background everytime you want to deblur an image when the goal is to have an algorithm near to real-time.

Once we have an estimation for the noise power, we compute the variance of the whole image to get the signal power and we divide the first value by the scond one, giving an estimation of the NSR.


\subsection{Blur metrics}

To compare different methods of deblurring, we need to be able to give the perceptual quality of an image. Suppose we have the original image $F$ and we want to estimate the quality of another image $\hat{F}$ that is a blurred version of $F$ or an approximation of $F$ after a deblurring operation. An easy way to do this is computing the Mean Square Error (MSE) between both images:
\begin{equation}
MSE=\frac{1}{MN} \sum\limits_{m=1}^{n=1}\left(F(m,n)-\hat{F}(m,n)\right)^2.
\end{equation}

The problem of this method is that it highly depends on the scale of intensity of the image. That's why we prefer the Peak Signal-to-Noise Ratio (PSNR) that avoid this problem:
\begin{equation}
PSNR = -10 log \frac{MSE}{S^2},
\end{equation}
where $S$ is the maximum value that a pixel can reach (255 for 8bits coding). Here the bigger the PSNR is, the better the quality of the image.

The problem of both methods is that we need the original image. That is only possible when we blur the images ourselves and we test our different deblurring methods. But in a real situation, we don't know $F$. The only image we have is the blurred image $G$ and the result of a deblurring method $\hat{F}$, so we can apply MSE and PSNR to those images. But we still only get an estimation of how close both images are, not of the blur intensity itself. Our next method determines the quality of an image objectively and without referential image. The technique that we found is a no-reference blur metric that is based on the analysis of the spread of the edges in an image. No-reference means that the metric is not relative to the original but is a absolute value associated to an image.

When an image is blurred, its high spatial frequency values in the spectrum are attenuated. Blur is perceptually apparent along edges so the technique is based on the smoothing effect of it on the edges. According to SOOOOOURCE (marziliano2002),it is sufficient to measure blur along vertical edgdes. Here is the algorithm:\\
First we detect the edges by a Sobel filter for example. Then we scan each row of the image. When we detect a pixel that corresponds to an edge, we look for the local maximum and minimum on that row. The distance (in pixels) between those extrema represents the local blur. Finally, we get the global blur measure for whole the image by averaging the local blur values over all edge locations.

The matlab function $bordSobel(I,meth,tresh,direction)$ implements this algorithm. The inputs are the processed image $I$, the method used to determine the edges ($1=Sobel, 2=Prewitt, 3=Canny$), a treshold for those methods that is fixed at $0.0215$ if it is not specified and finally the direction of the edges (vertical or horizontal).

%TODO résultats
%TODO changer nom de bordSobel car ca peut une autre meth que Sobel


\subsection{EdgeTaper}

One of the main characteristic of the frequency domain is its periodicity. It means that at the edges of the pictures \todo{To be continued after getting a nap} 

\subsection{RGB - grayscaled images : analysis and treatment}

Comme expliqué dans la partie de modélisation mathématique, les images NB se distinguent aux images de couleurs dans leur structure mathématique. La façon pour matlab de gérer l'espace des couleurs est d'adopter une représentation RGB de telle sorte que l'output pour un certain point du plan est un vecteur à 3 dimensions (one dimension per primitive colour) au lieu d'un scalaire dans le cas d'une image NB. L'image est donc représentée par une matrice à 3 dimensions et la couleur finale en un point correspond à une synthèse additive de ces trois composantes.

Dans le cas de notre travail, nous avons adapté le programme de telle sorte à ce qu'il fonctionne pour les deux types d'images (RGB ou NB). Il est également possible pour un utilisateur de choisir de convertir l'image à traiter en NB via l'interface graphique (qui exploite la fonction rgbtogray.m). L'avantage de convertir une image en NB avant de la traiter est le gain de temps (une dimension à traiter au lien des 3 de RGB). Par exemple le gain de temps en convertissant les images de la camera de surveillance peut être assez important. De plus, suivant l'utilisation voulue, il n'est pas toujours nécessaire d'avoir une image finale en couleur (par exemple pour le défloutage d'une plaque d'immatriculation). 
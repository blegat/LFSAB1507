\section{Estimation of the PSF}
In the spatial domain, we have
\[ g = h * f \]
so in the fourier domain
\[ G = H \cdot F \]
$H$ is a 2-D $\sinc$ which can be detected in
various ways.

\subsection{Angle estimation with Radon}
The idea of Radon is, for a certain angle $\alpha$,
integrate over the perpendicular lines of the line at
angle $\theta$ passing by $(0,0)$.
The value of each integral is a function of the
point at which the perpendicular line intersects the
line.

When $\theta+\frac{\pi}{2}$ is the angle of blur,
we integrate along the peeks of the $\sinc$.
The function at this angle can be dinstinguished
by the value of its max or variance which is
greater than the others.

However, we need to be careful to several things which
are
\begin{itemize}
  \item If the image is not squared, the angle found
    is not linked with the same formula to the angle
    of blue. We solve it by cropping the image for
    it to be squared.
  \item The perpendicular line is longer for angles
    like $\frac{\pi}{4}$ so we need to divide by
    the length of the line.
    We divide our radon transform by the
    radon transform of an unit image to solve
    this problem.
  \item When we do a fourier transform, the
    signal is supposed to be periodic.
    However, with the blur, the pixels at each
    sides of the borders influence the image.
    Since the image is rectangular, this
    problem only appears along $x$ and $y$ axis
    and that creates artifacts at $0$ and $\frac{\pi}{2}$.
    This can be solved by multiplying our image
    by a hann window before applying the
    fourier transform.
\end{itemize}

\subsection{Angle estimation with Gabor}

\subsection{Length with Cepstrum}
We can verify that in
\[ \log(\F(|\log(|g|)|)), \]
along the angle of blur,
there is a peek at the center and a second peek
at a distance of $L$ in units of the discretization.

A peek detection algorithm can then be applied
to find $L$.

The problem is that it depends on the angle
and is sensible to errors in the angle
estimation.

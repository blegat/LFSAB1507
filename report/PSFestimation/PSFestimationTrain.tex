\section{Estimation of the PSF}
In the spatial domain, we have
\[ g = h * f \]
so in the fourier domain
\[ G = H \cdot F. \]
Since $h$ is a window,
its fourier transform $H$ is
a 2-D $\sinc$ at the angle of blur
which can be detected in various ways.

In~\cite{krahmer2006blind},
Radon and Cepstrum methods are presented.
Radon is said to be more appropriate for angle
estimation and Cepstrum more appropriate for length
estimation while being quite competitive to Radon
for angle estimation when the length of the blur
is small.
However, \cite{oliveira2007blind} says that
\cite{krahmer2006blind} is wrong and by adapting
the way Radon is computed like it is done
in~\cite{oliveira2006implementation},
it can also be used with success for small lengths and
it doesn't need normalization
(we'll see in section~\ref{sec:rt} what
normalization is,
understanding it now is not necessary).
We have however not found its source code of
``exact Radon transform''.

\subsection{Angle estimation with Radon}
\label{subsec:Radon}
\subsubsection{Preliminaries}
The idea is to detect the $\sinc$ shape in $H$.
The problem of $G$ is that even
after making the image 0-mean,
it's value at the center are so huge that the others
are negligeable.
We therefore compute its $\log$ or rather the $\log$
of its absolute value to avoid undefined values.
There is still problems with very small values which
have a $\log$ that can be very far in the negative axis
so we compute $S = \log(1 + |G|)$.

\begin{myfig}{cameraman_F_diff}
  {Effect of the blur on the fourier transform}
  \mysubfig{cameraman_F}{$S$ for the cameraman.}{0.32}
  \mysubfig{cameraman_F_replicate_40-30}{$S$ for the cameraman with a blur of $(40, 30)$ with ``replicate''.}{0.32}
  \mysubfig{cameraman_F_circular_40-30}{$S$ for the cameraman with a blur of $(40, 30)$ with ``circular''.}{0.32}
\end{myfig}

In figure~\ref{fig:cameraman_F_circular_40-30},
we see the effet of the blur on $S$.
$S$ for the original image is in
figure~\ref{fig:cameraman_F}.
However, as mentioned in \cite{krahmer2006blind},
there is a problem at angles 0 and $\frac{\pi}{2}$ which
can be seen in figure~\ref{fig:cameraman_F_replicate_40-30}.
The reason is that when we do a fft,
the signal is supposed to be periodic.
However, since we blur and then take a rectangular image,
the pixels at each sides of the borders are influenced
by pixels out of the image.
If those pixels are not the same as the pixels at the other side
(as if the image were periodic),
$g$ will be like a blurred image at $(L,\alpha)$ plus
some problems at the borders of a rectangle.
Since those problems appear along $x$ and $y$ axis,
that creates artifacts at $0$ and $\frac{\pi}{2}$ in $G$.

Figure~\ref{fig:cameraman_F_circular_40-30}
and~\ref{fig:cameraman_F_replicate_40-30} are excellent
examples.
When we blur an image artificially, we need to extrapolates
the pixels beyond the borders. Two ways:
\begin{enumerate}
  \item[circular:] We consider an infinite image made of
    the same image repetitively.
  \item[replicate:] The pixel beyond the border is the
    same as the pixel at the border.
\end{enumerate}
We can see that there is no problem with
 figure~\ref{fig:cameraman_F_circular_40-30}
since a circular image is what the fft is expecting.
Therefore, fft recognizes exactly the periodic image
blurred by the PSF and there is no artifacts
at 0 and $\frac{\pi}{2}$.

Sadly, for a real blurred image, we cannot decide the
way it is blurred and the chance that the image is
periodic is very small.

\cite{krahmer2006blind} gives a solution for this
which is multiplying our image by a hann window before
applying the fourier transform like done
in the figure~\ref{fig:sagar-hann}.
This methods work very well but unfortunately,
we haven't seen it immediately so we have taken a lot
of time trying to find other solutions.

We have isolated the radon in
the listing~\ref{lst:angle_estimator} and
the listing~\ref{lst:robust_angle_estimator} contains
our solution which calls it with the image at different
angles.
It does the following for the different angles
(by default, it is 0 and $\frac{\pi}{4}$):
\begin{enumerate}
  \item Rotates the image at this angle.
  \item Takes a square in the image.
  \item Calls the listing~\ref{lst:angle_estimator} which
    returns the $\var$ for each angle between 0 and $\pi$.
  \item Sets borders to $\{0\}$, $\{\frac{\pi}{2}\}$
    and $\{\pi\}$.
  \item While the distance between the $\argmax$ and the
    closest border is less than a $\epsilon > 0$ chosen
    beforehand, we increase the border to this angle.
  \item If there is no border left, no angle is chosen.
\end{enumerate}
In that way, we ignore the peeks at $0$, $\frac{\pi}{2}$
and $\pi$ along with their influence on the angles
close to them.
We are sure to take an angle where there is a peek
which is not one of the 3 peeks at 0, $\frac{\pi}{2}$, $\pi$.

Once we have done that for all the angles,
we take the maximum of their respective $\var$.

Since we rotate the image in spatial domain and then crop
a square, the artifacts at 0, $\frac{\pi}{2}$ and $\pi$ are
not rotated with the $\sinc$.

Sadly, the rotation removes the clarity of the blur since
we need to interpolate to get back square pixels.
The hann window is therefore a preferred solution.

Lastly, \cite{krahmer2006blind} warns us that the angle
detected by radon will not be the same than the angle
we are looking for if the image is not squared.
Indeed, if it is not squared, the sampling frequency
will still be the same for both dimensions since
the pixels are squares but the period
in the spatial domain will not be the same.
The sample frequency on the frequency domain will therefore
be different, which means that ``pixels'' in the frequency
domain will not be squares.
They give a formula to make the translation.
Our approach was rather simpler.
If the image is not squared, square it by cropping the
longest dimension.

\subsubsection{The Radon transform (RT)}
\label{sec:rt}
The Radon transform, for a certain angle $\theta$,
is an integration over the perpendicular lines of the line at
angle $\theta$ passing by $(0,0)$.
The value of each integral is a function of the
point at which the perpendicular line intersects the
line.

More precisely, we define a new frame $(x',y')$
which is $(x,y)$ rotated by $\theta$ in counterclockwise.
We then define
\[ R_\theta(x') = \int_{y_0'(x')}^{y_1'(x')} f'(x',y') \dif y' \]
where $f'$ is the image in the new frame,
$y_0'(x')$ is the smallest of $y$ so that $f'(x',y_0'(x'))$ is in the image
and $y_1'(x')$ is the biggest such value.

\begin{myfig}{ones}
  {Analysis of the RT on an unit matrix (filled with ones)}
  \mysubfig{ones-Rad}{RT of an unit matrix.}{0.45}
  \mysubfig{ones-Var}{Variance and max for each angle of the RT.}{0.45}
\end{myfig}

The problem is that the perpendicular line is longer for angles
like $\frac{\pi}{4}$ at $x' = 0$.
We can see this in the figure~\ref{fig:ones-Rad},
this is a RT of an unit matrix so the value of the RT is
basically $y_1'(x') - y_0'(x')$.
At $(\ang{45},0)$, we can see that it is maximum.
What \cite{oliveira2007blind} suggests is to use a fixed
value for $y_0'$ and $y_1'$ but that's not
what the implementation of matlab does.
This is why he implements a new RT in~\cite{oliveira2006implementation}.

Since we've not found it's source code,
we'll just do a simple normalization
by dividing by the length of the line,
which is $y_1'(x') - y_0'(x')$ as
suggested by~\cite{krahmer2006blind}.
To do that easily,
we divide our RT pointwise by the
RT of an unit image.

Now that we have solved this normalization problem,
we can see the effect of the angle of the blur $\alpha$
on the normalized RT.

When $\theta = \alpha$,
$x'$ is perpendicular to the peeks of the $\sinc$ so
we integrate along the peeks of the $\sinc$.
For example, in the \figref{sagar-S}, we can see
that the peeks are at an angle of $\ang{9}$ since the
blur is at $\ang{171}$.
$R_\alpha$ can therefore be distinguished by the value
of its $\max$ or $\var$ which will be greater than the
others.
\cite{oliveira2007blind} suggests to use the $\var$
but we have found images where the $\max$ was more
accurate, mostly while we had not found the hann method.
With hann, the $\var$ performs almost always
better than the $\max$.

If we compute the $\var$, we must crop $(x',R_\theta(x'))$ when
this $x'$ is out of the image for some other angle.
We can see in the figure~\ref{fig:sagar-RadNorm} that
there is a lot more values at \ang{45} than for other angles.
It is therefore cropped to figure~\ref{fig:sagar-RadFocus}.

As we can see with the picture Sagar, if we had not normalized,
the $\var$ would have given us \ang{135}
(see figure~\ref{fig:sagar-Var}) while the $\max$
would still be correct
(see figure~\ref{fig:sagar-max}) but almost.
With a smaller blur (for Sagar, we have $L = 25$) we would
have found \ang{135} too or \ang{45}.

With the normalization, \figref{sagar-VarN} and
\figref{sagar-maxN} give both \ang{171}, which is the right
answer.

\begin{myfig}{sagar}
  {Analysis of the normalization of radon for Sagar which is an imaged blur at
  $(25,\ang{171})$.}
  \mysubfig{sagar-hann}{Square in the image multiplied by a hann window.}{0.4}
  \mysubfig{sagar-S}{$S$ where we can clearly see the $\sinc$.}{0.4}
  \mysubfigg{sagar-Rad}{RT of $S$.}{0.3}{trim=8cm 0cm 8cm 0cm, clip, height=4cm}
  \mysubfigg{sagar-RadNorm}{Normalized RT of $S$.}{0.3}{trim=8cm 0cm 8cm 0cm, clip, height=4cm}
  \mysubfigg{sagar-RadFocus}{Normalized and cropped RT of $S$.}{0.3}{trim=7cm 0cm 7cm 0cm, clip, height=4cm}
\end{myfig}
\begin{myfig}{sagar-plot}
  {$\var$ and $\max$ of the RT of $S$ for Sagar}
  \mysubfig{sagar-Var}{$\var$ of the RT.}{0.4}
  \mysubfig{sagar-VarN}{$\var$ of the normalized RT.}{0.4}
  \mysubfig{sagar-max}{$\max$ of the RT.}{0.4}
  \mysubfig{sagar-maxN}{$\max$ of the normalized RT.}{0.4}
\end{myfig}

\subsection{Angle estimation with Gabor}
\label{subsec:Gabor}
The Gabor filter is a gaussian filter modulated by a sinusoidal wave. The function of this filter is given by:
\begin{equation}
B(x,y)=\dfrac{1}{2\pi \sigma_x \sigma_y} exp\left[-\frac{1}{2}\left(\frac{x^2}{\sigma_x^2}+ \frac{y^2}{\sigma_y^2} \right) -j\omega(xcos\phi + ycos\phi)\right],
\label{filtreGabor}
\end{equation}
where $\sigma_x$ and $\sigma_y$ are the standard deviations in the $x$ and $y$ directions. The parameters $\phi$ and $\omega$ are the direction and the frequency of the filter respectively. Only $\phi$ will vary while the other parameters stay fixed. According to the article \cite{Dash20141634}, experimentation has shown that good values for $\sigma$ and $\omega$ are $3$ and $1.75$ respectively. The method consists in applying the Gabor filter to the power spectrum of the blurred image and then detecting the blur angle $\theta_{blur}$ by searching for the $\phi$ that gives the highest response value. This value can be calculated using $L_2$ norm. The $\phi$ obtained by this method is the blur angle. Here are the steps of the algorithm:
\begin{enumerate}
\item computation of the spectrum of the blurred image by a two dimensional Fourier transform
\item taking the logarithm of it
\item convolving this with the Gabor function given by equation $(\ref{filtreGabor})$ for different $\phi$, the result is $R(\phi)$
\item for every $\phi$, taking the $L_2$ norm: $||R(\phi)||$
\item the blur angle is the parameter $\phi$ that gives the biggest norm:
\begin{equation}
\alpha_{blur} = \arg \left\lbrace \max_{\phi}R(\phi)\right\rbrace.
\end{equation}
\end{enumerate}

This algorithm is implemented by our matlab function \texttt{angle\_estimator\_Gabor(f,thetamin, thetamax)}, where the input $f$ is the blurred image. The parameter $\phi$ will vary from $thetamin$ to $thetamax$. If those inputs are not specified, we set them to $0$ for $thetamin$ and $180$ for $thetamax$.


\subsection{Length with Cepstrum}
\label{subsec:Cep}
Let $L$ be the length of the blur.
\cite{biemond1990iterative} shows that in% \todo{clarifier la première phrase, benoît?}
\[ \F^{-1}\log(|\F(g)|), \]
there is negative at a distance $L$ from the origin.
The $\log$ transforms the convolution $h*f$ in the sum
$\F^{-1}\log(|\F(h)|) + \F^{-1}\log(|\F(f)|)$
(if we do not take the noise into account).
The peeks detected are the peeks of the inverse fourier
transform of the $\log$ of a $\sinc$.
\cite{krahmer2006blind} even proposes to use the inverse
tangent of the slope from the origin to this peek to
estimate the length but disadvice to do so,
arguing that Radon is much better at this job.
\cite{krahmer2006blind} however advises to use the Cepstrum
for length estimation.

\cite{Deshpande2014606} argues that this Cespstrum does
not work properly for small blur lengths or for high
frequency components in the original image which
are wrongly identified as the location of peeks.
He says that applying second $\log$ would enhance the $\sinc$ peeks
in comparison to the image frequencies.
%\todo{pas compris la derniere phrase benoît}

It's strategy is to search for peeks
along the angle of blur in
the following modified Cepstrum
\[ \log(\F(|\log(|\F(g)|)|)). \]
There should be a peek at the center and a second peek
at a distance of $L$ in units of the discretization.

A peek detection algorithm can therefore be applied
to find $L$.

The problem is that we have to search for the peeks along
the blur angle so this method will have
bad results if the angle is badly estimated.

\cite{Deshpande2014606} proposes to rotate the image
with $-\alpha$ in spatial domain so we only have to
search along the horizontal line in the modified Ceptrum.
That makes the method work for square images.
However, we can just estimate the length of the squared image.
If we do so, we can rotate the modified cepstrum which is less
likely to cause problems since we rotate \emph{after}
computing the Cepstrum.

To be robust to angle errors we could have taken the mean of
several lines besides the horizontal line.
But that would also make it less precise if we have exactly
the good angle.
We have not implemented this technique but it would
be easy to add.

\cite{Deshpande2014606} claims to have a peek detection
algorithm that does not require manual intervention.
Ours only need one parameter $k$.
The right value of $k$ depends on the image but is
usually 3 for small images and/or small blur,
5 for normal images and 8 for big images or big blur.
Our algorithm is the following
\begin{enumerate}
  \item If we try to find the max just after the central peek,
    we will just find the point just besides it because even
    if it is not a peek, it is high because it is in the
    influence of the central peek.
    Once we are in the influence of the central peek,
    the plot is going down.
    Therefore, we wait before we are not in the influence of that peek
    anymore, which is implemented by
    ``while $i \neq \argmax(i-k\text{ to }i)$'' so
    while we are not the max of the $k$ points before us
    (including us).
    If we are the max of those points,
    that means that we are kind of increasing for $k$
    points.
  \item When we are no more in its influence, we take the
    $\argmax$ of the rest of the modified Cepstrum.
\end{enumerate}

The secondary peek should be on both side of the peek at the
origin.
We take the one at the right.

To remove the need of choosing $k$, we could,
for each $k$, find the peeks on the left and the right and
find their distance with the origin.
We then take a $k$ such that the distance
the difference between left and right is minimum.
This is a suggestion of amelioration,
we have not implemented or tested it.

Let's illustrate this with the \figref{cameraman_cepstrum}.
For a angle of \ang{0}, we clearly see the two peeks
in the \figref{cameraman-plot-60_0}.
In the \figref{cameraman-contourf-60_0},
we see that those peeks are on the horizontal line.
For \ang{30}, we see in \figref{cameraman-contourf-60_30}
that the line is not horizontal anymore but it is at \ang{30}.
That's why we rotate our modified cepstrum by \ang{30}.

The peek is now clearly visible (see \figref{cameraman-plot-60_30}) but
there is a small sequence of 5 points for which the max is at a distance
14 and is still in the influence of the central peek.
We therefore need to take $k = 8$ to get the right length.

\begin{myfig}{cameraman_cepstrum}
  {Analysis of the modified Ceptstrum on the cameraman with a blur of $(60,\ang{0})$ and $(60,\ang{30})$.}
  \mysubfig{cameraman-plot-60_0}{Plot of the horizontal line for \ang{0}}{0.45}%
  \mysubfig{cameraman-contourf-60_0}{Modified Cepstrum for \ang{0}}{0.45}
  \mysubfig{cameraman-surf-60_0}{Modified Cepstrm for \ang{0}}{0.45}%
  \mysubfig{cameraman-plot-60_30}{Plot of the horizontal line for \ang{30}}{0.45}
  \mysubfig{cameraman-contourf-60_30}{Modified Cepstrum for \ang{30}}{0.45}
\end{myfig}

\chapter{General strategy}
\section{Train}
Let's summarize the model.
We know $g(x,y)$ and we know that there is a $h(x,y)$ such that
\[ g(x,y) = h(x,y) * f(x,y) + e(x,y). \]
We would like to find $f(x,y)$.
To obtain it, we need to deconvolve $g(x,y)$ with a method
robust to the noise $e(x,y)$.

However, we don't know $h(x,y)$ so we first need to estimate it
in a way that is also robust to the noise.
It is easier to estimate $h(x,y)$ if we make some assumptions.
For our problem, we have a motion blur which means that
$h(x,y)$ only depends on an angle and a length which are
respectively the angle at which the blur is and the number of
pixel on which each pixel depends.

Once we have an estimate of $h(x,y)$, we then deconvolve
$g(x,y)$ to get an estimate $\hat{f}(x,y)$.
For some deconvolution method, we need an estimate of
some properties of the noise which is done only using $g(x,y)$.

\section{Camera}
On the camera case, we have a serie of images.
The images are obtained from a fixed camera capturing a fixed
background with some objects appearing in some images.
\begin{enumerate}
  \item For the first images, we only estimate the background.
    The background is obtained as the mean of the pixels,
    we also calculate the variance.
  \item For each following images, we use the background and the variance
    to detect the foreground.
    We take the ``biggest'' connected shape and then deblur it.
    Once it is deblurred, we put it back on the background.
  \item We can then update the background.
    Indeed, if for example, a chair is added to the background
    or the light is changing, we need the background to be updated.
    The mean is then not exactly the mean nor the variance
    since we wand new images to have more importance than previous
    images.
\end{enumerate}

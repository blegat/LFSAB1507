\chapter{General strategy}

The goal of this chapter is to explain the main steps of our strategy of deblurring. As we mentioned in the previous chapter, there are two main steps:\\
1. Estimation of the Point Spread Function (PSF) \\
2. Deconvolution with the estimated PSF

We also give a brief description about our specific situations (train and camera) and about the extra thing we need to do for the security camera - situation as we only need to deblur a part of the image in that case. Then we give some explanation about our supplementary item: fast computing.

To estimate the PSF, we need more specifically to estimate the blur angle (in degrees) and the blur length (in number of pixels). We have two methods to estimate the angle, one that works, the other not at all. For the length estimation we only implemented one algorithm. On the opposite, the deconvolution part can be done by three different methods. 

Obviously, our methods work better for images that we blurred artificially, that means that we blurred them ourselves with our matlab function \texttt{blur.m}. We could easily verify our results as we knew which parameters of blur (angle and length) we set. But we also applied our algorithms on natural burred images. The results are less convincing but some are still good.
  
\section{Estimation of PSF}

The estimation of PSF works very well with artificially blurred images because they naturally meet the assumptions we made about how the image was blurred.
 
\subsection{Angle}
The method that works pretty well for the angle estimation is based on the Radon transform. The matlab function that implements this method is \texttt{robust_angle_estimator.m}. We tried to implement another method to estimate the blur angle but we didn't have the time to debug it. This method uses the Gabor filter and is implemented by our matlab function \texttt{angle_estimator_Gabor.m}. 

\subsection{Length}
After the angle has been estimated, we try to find the blur length. Our method method uses the concept of signal's cepstrum. The Matlab function implementing this method is \verb|length_estimator.m|. 

\section{Deconvolution}
The three algorithms are:

1. Richardson–Lucy. matlab function : \texttt{deconvlucy.m}

2. Wiener. matlab function : \texttt{deconvwnr.m}

3. Reguralization. matlab function :\texttt{deconvreg.m}

\section{Specific situations}

These steps are directly applied for pictures taken in a train by our matlab function \texttt{deblur.m}. We cannot do that for a picture taken by the security camera as we only want to deblur the foreground of it. So for this situation we first need to create a background image based on the sample of statistical images. As the background can change with time (new item, different brightness,...), we change it in a dynamic way. Then we detect the foreground thanks to this background image and we apply what we just described ont that part of the image. Deblurring an image of the security camera is done by our matlab function \texttt{deblur_cam.m}.

\section{Fast computing}

We fix as objective to have a deblurring time for one picture of about 1 second. To reach it we first need to optimize our implementation by reducing the number of loops and stuff like that. But we also crop an image when all the information of that image is not needed. This allow us to have an execution time that is about $20$ times faster than if we didn't do that. The matlab function that implements that cropping is \texttt{compression.m}.
\chapter{Background estimator for cam}

Dans le cas où le background de la caméra de surveillance n'est pas fournit, il est nécessaire de pouvoir l'estimer. En effet, il sera indispensable à la détection d'un foreground et donc à la résolution du problème posé. 
\\
Toutefois, nous pouvons supposer que nous disposons d'une série de $k$ images (qui sont de même dimension $M$x$N$ et en nombre de suffisant) prises par la camera. Attention, par souci de facilité on se limite au cas d'un signal bidimensionnel (grey-scaled image), le cas des images couleurs est similaire en considérant une dimension de plus pour le signal ! 
\\
Nous allons réaliser un traitement statistique pour chaque pixels situé en position ($m$,$n$) (avec $m = 1,\cdots,M$ et $n = 1,\cdots,N$) des $k$ images de la série.
\\
\\
Soit $I_i$ ($i = 1,\cdots,k$) la $i$-ème image de la série. Pour tout $m$,$n$, nous calculons le premier quartile ($Q_1 (m,m)$), la médiane ($Q_2 (m,m)$) et le troisième quartile ($Q_3 (m,m)$) de la série statistique $I_1 (m,n), I_2 (m,n), \cdots, I_k (m,n)$. 
\\
Le principe est à présent de calculer une moyenne $M(m,n)$ de la série sans tenir compte des valeurs qui s'écartent trop de la médiane (c'est à dire des pixels appartenant au foreground). Nous prenons donc en compte dans le calcul de la moyenne que les pixels des images appartenant à l'intervalle $[Q_2 (m,m) - Inter(m,n);Q_2 (m,m) + Inter(m,n)]$ où $Inter(m,n)$ est l'espace interquartile et vaut $Q_3 (m,m) - Q_1 (m,m)$.
\\
La matrice finale M des moyennes ainsi formée de dimension $M$x$N$ constitue une estimation valable du background !
\\
\\
La fonction Matlab réalisant ce traitement est DetectBackgroundColor.m qui prend en entrée une cellule de k images de couleurs (3-dimentionnal matrice) et qui renvoie une cellule de 3 matrices 3-dimentionnal : le background M (matrice des moyennes), la matrice du second moments et celle de la variance des séries statistiques pour chaque pixels.